
The idea is a generalization of usual approach for subsets of $\mathbb{R}^2$. We say a pre-ordered set $\Lambda$ is upward filtered if for each $\{a,b\}$ there is a $c$, such that $a \leq c$ and $b\leq c$. A net in $X$ is a function $x_{\alpha}$ from such $\Lambda$ to $X$. We know by proposition 1.5.2. Pedersen, that each net in a Hausdorff space converges to at most one point. Using mentioned proposition, each continuous function $f$ in a set $A$ has a unique extension $\tilde{f}$ on $\bar{A}$ when we take $f(a)$ as limit of $f(a_{\alpha})$. Using the same point-wise approach in solution of exercise 8, one can think of $g \circ f = h \circ f$ as $g|_{f(A)}=h|_{f(A)}$. Now if $f(A)$ is dense in $B$, the extension of $g|_{f(A)}$ is unique and hence $g=h$.

% 

% It seems typing the solutions in details is time wrecking, we can skip typing altogether or write partially as like hints.
% I omitted the details in the solution for the same reason mentioned above.
