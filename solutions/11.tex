
The idea is a generalization of usual approach for subsets of $\mathbb{R}^2$. We say a pre-ordered set $\Lambda$ is upward filtered if for each $\{a,b\}$ there is a $c$, such that $a \leq c$ and $b\leq c$. A net in $X$ is a function $x_{\alpha}$ from such $\Lambda$ to $X$. We know by proposition 1.5.2. Pedersen, that each net in a Hausdorff space converges to at most one point. Using mentioned proposition, each continuous function $f$ in a set $A$ has a unique extension $\tilde{f}$ on $\bar{A}$ when we take $f(a)$ as limit of $f(a_{\alpha})$. Using the same point-wise approach in solution of exercise 8, one can think of $g \circ f = h \circ f$ as $g|_{f(A)}=h|_{f(A)}$. Now if $f(A)$ is dense in $B$, the extension of $g|_{f(A)}$ is unique and hence $g=h$.


Suppose $B_{\tau}$ is Hausdorff. Take $B_{\sigma}$ as a refinement of $B_{\tau}$ such that $ B_{\sigma}$ has the metric $d$. Let $ u(-)=d(f(A), -) $ in $ B_{\sigma} $ and $ g = u, h = 2 u $ in $ B_{\tau}$. Take $C =\mathbb R^{+}$ with final topology induced by $\{g,h\}$. $g,h$ are continuous and $g|_{f(A)}=h|_{f(A)}$, that means $g \circ f = h \circ f$, but we have $ g \neq h$ where $g,h$ are read as $g,h: B_{\tau} = B \to C$. We have to show that $C$ a Hausdorff space. 

\textbf{Lemma.} Suppose $U$ is an open set in topological space $X$ and $X^{\prime}$ is a refinement of $X$. We have $\bar {U} = \bigcap_{i} E_{i}$ , where $E_{i}$ is a closed set containing $U$. It immediately follows from this, that closure of $U$ in $X^{\prime}$ is a subset of closure of $U$ in $X$.

Take $x,y \in C$ and two open neighborhoods $(a,b)$ and $(c,d)$ of $x,y$. If $b \leq c$, there is no need for further investigation. Suppose $c < b$, then $(c,b)$ is not empty in $\mathbb R ^+$, moreover, as mentioned above, $(c,b)^{-} \subset [c,b]$. Now $(a,b) \backslash (c,b)^{-}$ and $(c,d) \backslash (c,b)^{-}$ are disjoint open neighborhoods of $x,y$ and hence, $C$ with the mentioned final topology is a Hausdorff space.


% It seems typing the solutions in details is time wrecking, we can skip typing altogether or write partially as like hints.
% I omitted the details in the solution for the same reason mentioned above.
