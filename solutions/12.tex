A relation $R:A\rightarrow B$ is monic in \textbf{Rel} iff $p_0(R)=A$ and for no distinct $a,a'\in A$ there is a $b\in B$ such that $aRb$ and $a'Rb$. For one side assume that $R$ is monic. If $p_0(R)\neq A$ there is an $a\in A$ such that $a\not\in p_0(A)$. Consider the two relations $S=\{\langle 0,a\rangle\}$ and $\emptyset$ between $\{0\}$ and $A$. It is clear that $R\circ S=\emptyset=R\circ\emptyset$, while $S\neq\emptyset$. For the second condition, assume that there are $a,a'\in A$ such that for some $b\in B$, we have both $aRb$ and $a'Rb$. Consider the two relations $S=\{\langle 0,a\rangle\}$ and $T=\{\langle 0,a'\rangle\}$. It is clear that
$R\circ S=\{\langle 0,b\rangle\}=R\circ T$, while $S\neq T$.

For the converse assume that $R$ satisfies the conditions stated above, and let $S,T:C\rightarrow A$ be two distinct relations. Without loss of generality we can assume that for some $c\in C$ and $a\in A$ we have $cSa$ but not $cTa$. By the first condition there is a $b\in B$ such that $aRb$ and by the second condition $a$ is the unique member of $A$ which is related to $b$ by $R$. So we have $c(R\circ S)b$ but not $c(R\circ T)b$, and hence $R\circ S\neq R\circ T$.