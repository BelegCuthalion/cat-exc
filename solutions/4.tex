It is trivial that in \textbf{Set} the bijections are isomorphisms. For the converse, assume that for $f:A\to B$ and $g:B\to A$, we have $gf=id_A$ and $fg=id_B$. The morphism $f$ must be surjective. If not there is a $b\in B$ that is not in the range of $f$. Such an $f$ cannot be in the range of $fg$ too, and hence $fg\neq id_B$. It also needs to be injective. If not there are $a,a'\in A$ such that $a\neq a'$ and $f(a)=f(a')$. Then we have $gf(a)=gf(a')$, and hence $gf\neq id_A$.

The following morphism in \textbf{Posets} is bijective, though not an isomorphism:

\[\begin{tikzcd}
	\bullet &&& \bullet \\
	\\
	\\
	\bullet &&& \bullet
	\arrow[from=4-4, to=1-4]
	\arrow[from=1-1, to=1-4]
	\arrow[from=4-1, to=4-4]
\end{tikzcd}\]

In \textbf{Posets} the isomorphisms are bijective homomorphisms whose inverses are also homomorphisms. That such morphisms are isomorphisms is clear. For the converse, assume that for $f:A\to B$ and $g:B\to A$ we have $gf=id_A$ and $fd=id_B$. By our argumentation in the case of \textbf{Set} it is clear that $f$ must be bijective. Moreover it is clear that the inverse of $f$, namely $g$, is a poset homomorphism.
